\documentclass[11pt]{article}

\pagestyle{plain}
\title{Rapport 2 : Décomposition LU et moindres carrés}
\author{Loïc Jabiro KAYITAKIRE}
\date{NOMA : 53272100}

\begin{document}
\maketitle

\section*{Question 1}

\subsection*{Définition}
Le conditionnement est un concept qui permet de mesurer la sensibilité d'un problème de calcul numérique par rapport à une perturbation. 
Il est mesuré grâce au nombre de conditionnement $ \kappa $ qui se calcule comme suit :
$$ \kappa(x) = \lim_{\delta \to 0}   \sup_{||\delta x|| \leq \delta}   \frac{\frac{||\delta x||}{||x||}}{\frac{||\delta A||}{||A||}} $$

\subsection*{Conditionnement par rapport à A}

Le problème de base est : $A^T A x = A^T B$
\\
On pose $A^T A = A'$ et $A^T B = B'$
\\
On suppose une perturbation $ \delta A' $ sur A'. On a alors le problème perturbé suivant : 
$$ (A' + \delta A') (x + \delta x) = B' $$
$$ A'x + A'\delta x + \delta A'x + \delta A' \delta x = B' $$
$ A'x = B'$ et $ \delta A' \delta x$ est négligeable par rapport au reste. On a alors :
$$ A'\delta x + \delta A' x = 0 $$
Ce qui donne ensuite :
$$ \delta x = -A'^{-1} \delta A' x $$
On a alors :
$$ ||\delta x|| = ||-A'^{-1} \delta A' x|| \leq ||A'^{-1}|| ||\delta A'|| ||x|| $$
Et donc :
$$ \frac{||\delta x||}{||x||} \leq ||A'||||A'^{-1}|| \frac{||\delta A'||}{||A'||} $$

Ainsi, $ \kappa(A^T A) = ||A^T A||||(A^T A)^{-1}|| $ 

En suivant un raisonnement similaire pour une perturbation sur B, on obtient :
$$ \kappa(A^T B) = ||A^T B||||(A^T A)^{-1}|| $$

\section*{Question 2}

La complexité est O(n^3) pour la décomposition LU. L'algorithme n'est pas parallélisable car les calculs dépendent des résultats précédents.
\section*{Question 3}

\section*{Question 4}

\end{document}