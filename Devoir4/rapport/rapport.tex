\documentclass[11pt]{article}
\pagestyle{plain}

\usepackage{amssymb}
\usepackage{amsmath}
\usepackage{amsfonts}
\usepackage{graphics}
\usepackage{graphicx}
\usepackage[margin=1in]{geometry}
\usepackage{svg}
\usepackage{float}
\usepackage{hyperref}

\title{LINMA1170 - Devoir 4\\Détection de communautés à l'aide de la SVD}
\author{Giovanni Karra - 45032100\\Pierre Leboutte - 69422100}

\begin{document}

\maketitle

Dans le cadre de ce projet, nous allons expliquer ce qu'est la décomposition SVD (Singular Value Decomposition) d'une matrice, et comment l'appliquer pour identifier les communautés dans les graphes.

\section*{SVD, kesako?}

\subsection*{Quelques concepts}
Pour comprendre la décompostion SVD, il est essentiel de connaitre certains concepts clé de l'algèbre linéaire.\\
\begin{itemize}
    \item Soit une matrice $A$, sa $matrice ~conjugu$é$e$ $A^*$ est équivalente à la la matrice $A^T$ dont on conjugue les entrées complexes. 
    \item Une matrice $A$ est dite $unitaire$ ssi $AA^* = I$.
\end{itemize}

\subsection*{La décomposition}
Le but de la décomposition SVD est de séparer une matrices en ses principales composantes.
\begin{align*}
    A = U \Sigma V^*
\end{align*}
où $A$ est une matrice $m\times n$ quelconque, $U$ est une matrice unitaire $m\times m$, $V$ est une matrice unitaire $n \times n$, et $\Sigma$ est une matrice diagonale $m \times n$.

\section*{Détection de communautés}
La détection de communautés c'est blablabla.

\subsection*{Utilisation de la SVD}
On fait SVD sur truc parce que ça fait truc haha on est trop fort.

\subsection*{Résulats}
Regardez à quel point notre algo est bon/merdique.

\end{document}