\documentclass[11pt]{article}
\pagestyle{plain}

\usepackage{amssymb}
\usepackage{amsmath}
\usepackage{amsfonts}
\usepackage{graphics}
\usepackage{graphicx}
\usepackage[margin=1in]{geometry}
\usepackage{svg}
\usepackage{float}

\title{LINMA1170 - Devoir 3}
\author{Giovanni Karra - 45032100}

\begin{document}

\maketitle

\section*{Questions théoriques}

\subsection*{Question 1}

Soit une matrice normale $A$ ($A^*A = AA^*$), montrons que dans sa décomposition de Schur ($A = QTQ^*$, avec $Q$ unitaire et $T$ triangulaire supérieure), $T$ doit être une matrice diagonale. Comme toute matrice carrée a une décomposition de Schur, nous pouvons écrire :
\begin{align}
	A = QTQ^* \Leftrightarrow Q^*AQ = T\\
	TT^* = Q^*A\underbrace{QQ^*}_IA^*Q = Q^*AA^*Q\\
	T^*T = Q^*A^*\underbrace{QQ^*}_IAQ = Q^*A^*AQ
\end{align}
Comme $A$ est normale, nous savons que $TT^* = T^*T$, avec $T$ triangulaire supérieure ($U$) et $T^*$ triangulaire inférieure ($L$). Nous avons donc:
\begin{align*}
	UL &= LU
\end{align*}
En partant de cette égalité, nous pouvons montrer itérativement que $T$ est diagonale.
\begin{align*}
	(UL)_{11} = \sum_{k=1}^{n}|t_{1k}|^2 &= |t_{11}|^2 = (LU)_{11}\\
	\Leftrightarrow \sum_{k=2}^{n}|t_{1k}|^2 = 0 &\Leftrightarrow t_{1k} = 0 ~~~ \forall k \in \{2...n\}\\
	(UL)_{22} = \sum_{k=2}^{n}|t_{2k}|^2 &= |t_{22}|^2 + \underbrace{|t_{12}|^2}_{=0} = (LU)_{22}\\
	\Leftrightarrow \sum_{k=3}^{n}|t_{2k}|^2 = 0 &\Leftrightarrow t_{2k} = 0 ~~~ \forall k \in \{3...n\}\\ . \\ . \\ . \\
	(UL)_{ii} = \sum_{k=i}^{n}|t_{ik}|^2 = \sum_{k=1}^{i}|t_{ki}|^2 &= |t_{ii}|^2 + \underbrace{\sum_{k=1}^{i-1}|t_{ki}|^2}_{= 0} = (LU)_{ii}\\
	\Leftrightarrow \sum_{k=i+1}^{n}|t_{ik}|^2 = 0 &\Leftrightarrow t_{ik} = 0 ~~~ \forall k \in \{i+1...n\}
\end{align*}
Nous pouvons conclure que $T$ est bel et bien diagonale.\\
Montrons maintenant que si $T$ est diagonale dans une décomposition de Schur, alors la matrice d'origine $A$ est normale. Nous commençons avec:
\begin{align}
	AA^* &= QTT^*Q^*\\
	A^*A &= QT^*TQ^*
\end{align}
Comme $T$ est diagonale, nous savons que $TT^* = T^*T$, car dans les deux cas nous avons $elem_{ij} = 0$ si $i \neq j$, et $elem_{ij} = |t_{ij}|^2$ sinon. Nous pouvons donc dire que $QTT^*Q^* = QT^*TQ^*$, et donc que $AA^* = A^*A$.

\subsection*{Question 2}
Appliquer les algos que d'un côté + moins de mults sur hessenberg + Rayleigh

\subsection*{Question 3}
Jsp trop s'il faut utiliser la version full calc ou bien un truc plus formel du livre ou quoi

\section*{Analyse de performances}

\end{document}