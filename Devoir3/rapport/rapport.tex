\documentclass[11pt]{article}
\pagestyle{plain}

\usepackage{amssymb}
\usepackage{amsmath}
\usepackage{amsfonts}
\usepackage{graphics}
\usepackage{graphicx}
\usepackage[margin=1in]{geometry}
\usepackage{svg}
\usepackage{float}

\title{LINMA1170 - Devoir 2}
\author{Giovanni Karra - 45032100}

\begin{document}

\maketitle

\section*{Question 1}
Soit une matrice normale $A$ ($A^*A = AA^*$), montrons que dans sa décomposition de Schur ($A = UTU^*$, avec $U$ singulière), $T$ doit être une matrice diagonale. Comme toute matrice carrée a une décomposition de Schur, nous pouvons écrire :
\begin{align}
	A = UTU^* \Leftrightarrow U^*AU = T\\
	A^*A = UT^*\underbrace{U^*U}_{=I}TU^* = UT^*TU^*\\
	AA^* = UTU^*UT^*U^* = UTT^*U^*
\end{align}
Comme $A$ est normale, nous avons que :
\begin{align*}
	U
\end{align*}

\end{document}